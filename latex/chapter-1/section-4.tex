\section[Eastern European Views on Capitalism and Socialism]{Breakdown of Eastern European Views on Capitalism and Socialism (By Nation)\\\attribute{flesh\_eating\_turtle}{18th of December 2019}}
\subsection*{Introduction}

Hello comrades.
I decided to do some calculations to figure out the percentage of Eastern Europeans who feel that life in their country was better under socialism.
To do this, I made a table, and added the nations, populations, and degree of popular opinion within the individual nation.
My sources for these numbers and percentages are at the end.

\subsection*{Table and Analysis}

\begin{center}
\begin{tabular}{|l|l|l|l|}
\hline
\multirow{2}{*}{\textbf{Nation}} & \multirow{2}{*}{\textbf{Population}} & \multicolumn{2}{|c|}{\textbf{Favoring Life Under Socialism}} \\\cline{3-4}
& & \textbf{Percent} & \textbf{Number} \\\hline
Russia & 144,438,554 & 63\% & 95,329,446 \\\hline
Ukraine & 42,386,403 & 72\% & 30,518,210 \\\hline
Belarus & 9,491,800 & 38\% & 3,606,884 \\\hline
Moldova & 2,681,735 & 42\% & 1,126,328 \\\hline
Azerbaijan & 9,981,457 & 31\% & 3,094,252 \\\hline
Georgia & 3,729,600 & 33\% & 1,230,768 \\\hline
Armenia & 3,046,100 & 66\% & 2,010,426 \\\hline
Kazakhstan & 18,195,900 & 25\% & 4,548,975 \\\hline
Tajikistan & 9,420,175 & 52\% & 4,898,491 \\\hline
Kyrgyzstan & 6,389,500 & 61\% & 3,897,595 \\\hline
Turkmenistan & 5,983,043 & 8\% & 478,643 \\\hline
Lithuania & 2,742,221 & 30\% & 822,666 \\\hline
Bulgaria & 7,000,039 & 68\% & 4,760,027 \\\hline
Serbia & 7,020,858 & 81\% & 5,686,895 \\\hline
Romania & 20,121,641 & 69\% & 13,883,932 \\\hline
Hungary & 9,798,000 & 45\% & 4,409,100 \\\hline
Eastern Germany & 13,600,000 & 20\% & 2,600,000 \\\hline
Poland & 37,868,701 & 19\% & 7,195,053 \\\hline
Czech Republic & 10,698,355 & 22\% & 2,353,638 \\\hline
Slovakia & 5,458,230 & 39\% & 2,128,710 \\\hline
\textbf{Total} & \textbf{370,052,312} & \textbf{53\%} & \textbf{194,580,039} \\\hline
\end{tabular}
\end{center}

It is clear from these statistics that the capitalist system has failed to improve life for the majority of people in Eastern Europe.

It must also be noted that these numbers factor in the entire population of these nations, including the young people born into capitalism.
However, as Pew Research Center notes,\footcite[p. 29]{pew} ``those who lived through communism have a more negative view of the post-communist era.''
In other words, the percentage favoring socialism would be significantly higher if those who never actually experienced socialism were discounted.
This would also be true if we discounted those described by Pew as ``higher income,'' as among the working class, support for socialism is far higher (as Pew shows in their study).
For example, in Slovakia, 67\% of ``higher income'' people say that capitalism has improved their standard of living, while only 37\% of ``lower income'' people say the same.
However, given that most studies do not make this distinction, I have included the entire populations here.

One might also note that support for socialism tends to be highest in those nations which had their own independent revolutions.
Nations which were compelled to adopt socialism (such as the Czech Republic) tend to have less favorable views.
This indicates the importance of independent revolution.

Finally, it should be noted that support for socialism in Belarus is likely higher than the statistic indicates.
This is because that nation has retained a largely state-owned economy (as well as extensive social benefits) since the fall of the USSR, and thus the question asked in Belarus (``Has the fall of the USSR hurt your country?'') is likely to have given a misleading answer (they were not as hurt because they largely retained socialism).

\subsection*{Question Details and Conclusion}

In some nations, the people were asked ``Has the fall of the USSR benefited or harmed your country?'' In these cases, the percentage that said ``harmed'' was counted. In other nations, the people were asked whether capitalism has had a positive impact on living standards. In these cases, the percentage that said ``no'' was counted. In Serbia, people were asked when they lived best. Those who chose ``under socialism'' were counted.

For those nations not included here, I was unable to find the statistics. I apologize for any omissions.
\nocite{*}
