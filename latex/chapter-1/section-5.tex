\section[Debunking the ``Resources for Anti Communists'']{Debunking the ``Resources for Anti Communists''\\\small{\attribute{ImNotMarshalZhukov}{29th of December 2017}, edited by \userfmt{crazy01010}}}
\subsection*{Preface}
Recently I stumbled upon the ``resources for anti communists'', \href{https://www.reddit.com/r/EnoughCommieSpam/wiki/anti_commie_resources}{found here} (or as they spell it, rersources [since fixed - Ed.]) by the folks over at \href{https://np.reddit.com/r/enoughcommiespam}{r/enoughcommiespam}, and figured since I had some time to kill, I would whip this up.
Its just about exactly what you would expect, irrelevant or misconstrued data nearly three hours and a lot of research later, I have come up with this: Comrades, behold \emph{debunking resources for anti-communism}.

\subsection{``The Communist Manifesto''}
I don't know why this is here, but alright. Maybe I misjudged this list?

\subsection{2 fiction novels (Solzhenitsyn)}
No, didn't misjudge it.
No comment.
Using fiction to attempt to delegitimize a political ideology is ridiculous.
Especially fiction by an anti-Semitic monarchist.
This so called ``first hand account'' is nothing more than a collection of politically convenient rumours.
I think I'll take his wife's word for it.\footcite{natalya}
\begin{quote}
    Natalya Reshetovskaya, Aleksandr Solzhenitsyn's first wife, wrote in her memoirs that ``The Gulag Archipelago'' was based on ``campfire folklore'' as opposed to objective facts.
    She wrote that she was perplexed that the Western media had accepted The Gulag Archipelago as ``the solemn, ultimate truth'', saying that its significance had been ``overestimated and wrongly appraised''.
    She said that her husband did not regard the work as ``historical research, or scientific research.''
\end{quote}
This particular claim was admittedly made under somewhat suspicious circumstances, but it does aptly describe the Gulag Archipelago.
Not only are its anecdotes unverifiable, but the interspersed facts (using the term lightly) Solzhenitsyn gives about the gulag is absolutely ridiculous.
He claims that 60 million people (i.e. 1/3 of the population) were interred over its existence.\footcite{solzhenitsyn}
At the average gulag mortality rate of 5.34\%,\footcite{remembering} that means that, assuming equal distribution of gulag inmates from the 1930--1953 period, 3.5 million people would have died in the gulag.
The truth is that less than 1/3 of this many people died or were imprisoned.\footcite{penal-system}
Compared with other prison systems of the day and age, it is less then exceptional.
Interestingly, the post-war gulag system, with an average mortality rate of 0.725\%,\footcite{remembering} is somewhat comparable to the \emph{modern} Russian prison system that has a mortality rate of 0.56\%.\footcite{russian-prison}
Even given nearly 70 years of medical advancements, the difference in mortality between post-war Soviet Gulags and modern Russian prisons is only 0.165\%.
The American prison mortality rate of 0.225\% isn't all that much lower,\footcite{prison-mortality} and despite being the richest nation on earth and having 68 years of medical and technological development, there is only a 0.5\% difference between it and the so-called ``most gruesome prison system in history.''

\subsection{``The Great Terror: Stalin's Purge of the thirties'' - Conquest}
I don't think I need to explain why this one is bunk to this audience, but never the less one part of the summary [in \href{https://www.reddit.com/r/EnoughCommieSpam/wiki/anti_commie_resources}{the list}] stands out:
\begin{quote}
    When Soviet archives were opened following the collapse of the Soviet Union, and many of Conquest's claims were validated, Kingsley Amis joked that the book should be re-titled ``I Told You So, You Fucking Fools''.
    A revised edition titled ``The Great Terror: A Reassessment'' was printed in 1990.
\end{quote}
A quick note about this: the revised edition came out in 1990 (read: one year before the dissolution of the USSR) and just about all of the claims made by Conquest in both his original book and the revised edition turned out to be at best heavily inflated and at worst totally false.
A good example of this is his death toll of Holodomor, the supposedly man-made famine that ravaged the USSR in 1932-1933.
He puts the death toll of the famine at ``at least 7 million'' however, released archives show a death toll of 2.4 million, and this claim is substantiated by a demographics analysis done in 2010 (Snider, 2010), but it has been discounted because, and I quote Timothy Snider ``it too closely matches the official soviet figures.''
\emph{Gee, I wonder why.}
These archives were also considered state secrets throughout the USSR's existence, making it highly unlikely that they, along with other data from the time period, were tampered with or falsely recorded.

\subsection{``Origins of the Great Purge'' by J. Arch Getty}
%A book the author of this list clearly didn't read, considering they said the book claims over a million people were killed in the great purge (spoiler: it says fewer than 700,000) This one really belongs more in our masterpost, as it disproves much of the other work presented

\subsection{``Soviet Psychiatric Abuse: A shadow over world psychiatry''}
%A book about the use of psychiatry as a ``tool for political repression in the USSR under Brezhnev'' (less than 125 people were ``repressed'' using psychiatry from 1960 to the collapse of the soviet union(Buyanov, 1993)) Il quote Russian Psychiatrist Mikhail Buyanov ``among the persons were many fanatical nationalists, religious sectarians, and political paranoiacs who after escaping to freedom corrupted the masses, сrammed their heads with nonsense, carried away immature people with their ideas through the connivance of the so-called progressive intelligentsia, and a result of it is wars, blood, and reciprocal hatred''

\subsection{``Behind the Urals'' – John Scott}
%This one is interesting, because they have attempted to turn a pro-soviet book on its head, and in the process failed miserably. Every country undergoing industrialization has had some period in which living standards have been quite poor for the working class(some photos]. In the west, this period served to enhance the profit-margin or capitalists, but in the Soviet Union it served a different purpose. The Soviet Union had found itself surrounded by enemies, and it felt the pressing need to industrialize in order to better compete with its rivals and raise living standards for the population. The Russian empire in 1913 had approximately 3.5-4.5 million industrial workers in a country of approximately 190 million peoplesource, and around 85 percent of the population lived in rural areas. Over the course of the first five-year plan, despite the massive gains in productivity, living standards remained fairly stagnant, as the primary focus was on heavy industry. However, by the late 1940s living standards had improved, and goods were no longer being rationed. The struggle of the 1930s was comparable to the struggle of British industrial workers from the 1750s to the outbreak of world war two, the only difference being that the soviet struggle lasted a decade (more if you include world war two and its aftermath) and the British struggle lasted two centuries. Not only this, but workers were enthusiastic about their work (as the book shows), and had quite a few elements of self-management.

\subsection{``The Lysenko affair'' – David Joravsky}
%The Lysenko incident for those unfamiliar was the official promotion of Lysenkoism, an agricultural pseudoscience promoting something similar to Lamarckian inheritance. The Lysenko affair is admittedly a stain on soviet science, but in the context of the Soviet Union in the 1930's and 40's, it made (slightly) more sense. As you all well know, the 1930s brought on an industrial advancement in the Soviet Union that has never before or since been rivaled. In four years, industrial production doubled and agricultural production skyrocketed with collectivizationsource. Particularly in agriculture, the Soviet authorities were looking to increase agricultural productivity. They made a decision to support Lysenko in his theories of rejecting conventional genetics, and made use of techniques like deep plowing. And they seemed to work. During the 1930s the USSR saw a huge uptick in agricultural productivity, but it was only a few decades later that it was realized it was not because of Lysenkoism, but in spite of it. Around the same time Lysenko's ideas began to be put into practice, collectivization and the mechanization of agriculture took place, which is what triggered the spike in yields. Lysenko continued to insist that his ideas were a factor in rising agricultural productivity, and often attacked geneticists for not contributing to soviet agriculture. It did not help that he was an extremely shrewd statesman and knew how to navigate soviet politics. From a scientific perspective, we must remember that this was over 20 years before the discovery of DNA, and the soviet leadership was rather preoccupied with achieving what effectively amounted to one hundred years of industrial progress in less than 5 years. It is also implied that the soviet leadership was inflexible in their attitudes towards science, but this is not the case. Arnold Chikobava, a soviet linguist once wrote a letter to Stalin to inform him that the prevailing linguistic theory in the Soviet Union at the time, dubbed Marxist Linguistics was incorrect, and that he believed that the he should reconsider his policy. Stalin invited Arnold to a dinner that lasted from 9PM to 7AM in which Arnold explained the basics of linguistics and presented his evidence(Montefiore, 2003). Stalin then summoned the Politburo to debate the matter and announced that he had changed his mind on the subject. Considering nationalities (of which linguistics plays a major role) was one of his most notable specialties academically, to change his mind on a topic so radically important to his work demonstrates that he was much more open-minded than most think, and thus it is unlikely that there was widespread knowledge that Lysenkoism was bunk. In conclusion, lysenkoism was a mistake, but it was not one that was well known to the leadership at the time

\subsection{``Einstein and Soviet Ideology'' - Alexander Vucinich}
%This criticism of the Soviet Union is vastly undercut by one crucial fact: there was never suppression of physics-related research in the Soviet Union. Soviet institutes questioned general relativity on some occasions, and some individual scientists promoted a statistical interpretation of general relativity, but never was research into physics suppressed. ``Contemporary cases of the rejection of mainstream physics research by Marxists still exist'' In short, no they don't. They simply define the contradiction as the appearance of the phenomenon and the reality of the phenomenon.

\subsection{``The Ghost of the Executed Engineer'' - Loren Graham}
%Throughout my wading through this list, I have become more and more convinced that the makers of it didn't read the books they are recommending, or at the very least did a very poor job at understanding them. The Ghost of the Executed engineer is a fictionalized account of the life of engineer Peter Palchinsky that uses him as a plot device to criticize soviet industrialization. I don't know the specifics of Palchinsky's life or the trial that led up to his death, so I will refrain from commenting, but the experiences of one man are in no way representative of the 170 million people living in the Soviet Union during the historical setting of this book. Another interesting point is that the timeframe in which Palchinsky's story takes place was during the late NEP era, therefore I would be cautious about using it to represent the entire soviet economy

\subsection{``Animal Farm'' - George Orwell}
%This is obviously a fiction book, as you well know, but I felt that it deserves a special category of its own, as it is so often mentioned by anti-communists as a brilliant work. There are a whole host of problems associated with using fiction novels in political discourse, but the one I will be touching on in this piece is simply the fact that Orwell was talking out of his ass, for lack of a better term. Orwell never once set foot in the USSR, nor did he ever do any in-depth research as to what life was really like in the Soviet Union. A self-proclaimed democratic socialist, later in life he even betrayed his fellow ``comrades'' and wrote a list of potential communists to MI5 . The book was nothing more than an anti-Soviet polemic designed to sell copies in the west and annoy communists in the east. Furthermore, many of the allegories in the book are simply under/overrepresentations of the soviet system. A good example is the construction of the wind mill, which is supposed to represent soviet industrialization. A more accurate picture would be if they built 5 windmills, doubled the farm yield and moved the animals from barns to apartments (a quick note, the previous statement was just to make a point, pay no attention to specific numbers.) Furthermore, the representation of soviet ``totalitarianism'' is vastly overblown (see-origins of the great purge by J-Arch Getty and Another view on Stalin by Ludo Martens). In conclusion, Animal farm deserves no more a place in serious discussions about communism than American Sniper belongs in discussion about capitalism.

\subsection{``The Killing Wind'' - Tan Hecheng}
%This book details the Daoxian massacre, a massacre by red guard groups during the cultural revolution. There is no doubt that this massacre was unfortunate, but the actions of individual red guard groups acting on their own accords can hardly be blamed on chairman Mao or even Mao in general. I would also recommend checking out the PRC section in the debunking anti-communism masterpost found here

\subsection{Debunking Marxism-101}
%I will admit, it is somewhat impressive that someone took the time to compile a document like this. However, it appears to be based on flawed assumptions and neoliberal analysis. After some searching I found a video by TheFinnishBolshevik that tackles this very analysis here (https://www.youtube.com/watch?v=h7oT-SIjXoI&ab_channel=TheFinnishBolshevik)

\subsection{The revolution will not be adequately sourced}
This one attempts to debunk the Debunking anti-communism masterpost. This one has already been covered by Kyle Joseph \href{https://www.reddit.com/r/communism/comments/3lot78/bad_history_indeed_in_defense_of_the/}{here.}\footnote{\url{https://www.reddit.com/r/communism/comments/3lot78/bad_history_indeed_in_defense_of_the/}}

\subsection*{In Conclusion}
%I think most reading this have already come to this conclusion, but the ``resources for anti-communists'' is not a valuable pool of resources, but rather a collection of misinterpreted and flawed documents, mixed in with some that don't belong in the first place and the requisite wasn't real communism strawman. Rightists, you have really outdone yourselves
