\section[Socialism is Good For You]{Socialism is Good For You: Health, Welfare, and Quality of Life\\\small{Written by /u/flesh\_eating\_turtle on the 1st of March 2020}}
\subsection*{Introduction}

Hello everybody.
When discussing a contentious political issue, it is often useful to examine the empirical evidence before coming to a conclusion.
Seeing as healthcare is consistently ranked as one of the most important issues in nations like the USA,\footcite{hyrn-issues} it will be helpful to examine the matter more closely, to determine what socialism has to offer here.
Feel free to use \href{https://scihub.bban.top/}{Sci-Hub} to bypass any paywalls.

\subsection*{Socialism, Health, and Welfare}

According to a study by Vicente Navarro (Johns Hopkins University),\footcite{navarro-socialism} published in the \textit{International Journal of Health Services}, ``contrary to dominant ideology, socialism and socialist forces have been, for the most part, better able to improve health conditions than have capitalism and capitalist forces.''
He states that "the historical experience of socialism has not been one of failure.
To the contrary: it has been, for the most part, more successful than capitalism in improving the health conditions of the world's populations."

A well-known study published in the \textit{American Journal of Public Health} found that ``socialist countries generally have achieved better PQL [physical quality of life] outcomes than the capitalist countries at equivalent levels of economic development.''\footcite{cereseto-economic}
These results were verified in a later follow-up study,\footcite{lena-political} published in the \textit{International Journal of Health Services}, which found that ``in general, nations with strong left-wing regimes have more favorable health outcomes (e.g., longer life expectancies and lower mortality rates) than do those with strong right-wing regimes.''

Nobel-winning economist Amartya Sen (Harvard University) authored a study looking at quality of life in developing countries.\footcite{sen-public}
He found that ``Clearly the relative performance of communist countries is superior,'' prompting him to remark, ``One thought that is bound to occur is that communism is good for poverty removal.''
Similarly, a study published in the journal \textit{Population and Development Review} observed ``a general association between communism and low mortality, at least among poor countries.''\footcite{bryant-population}

Even reformist policies (insufficient though they are) can have a positive effect.
One study from Texas A\&M University found that ``citizens find life more rewarding as the generosity of the welfare state increases,''\footcite{pacek-welfare} concluding that ``socialism... provides the potential for improving the human condition, in so far as we agree that making 'life as satisfying as possible' is the appropriate standard of evaluation.''

Another study, published in the \textit{International Journal of Health Services},\footcite{navarro-political} found that ``political traditions more committed to redistributive policies (both economic and social) and full-employment policies, such as the social democratic parties, were generally more successful in improving the health of populations.''

\subsection*{Capitalism's Harmful Impact}

There is significant evidence that capitalist policies have a detrimental effect on health, particularly as they result in inequality.
According to a study published in the \textit{International Journal of Health Services},\footcite{ferre-inequalities} ``there is a strong correlation between income inequality and [negative] health outcomes.''
In addition, they found that ``countries that do not use International Monetary Fund loans perform better on health outcomes.''

\subsection*{Conclusion}

In short, socialism provides the best means of achieving high quality-of-life and good health outcomes, compared to capitalism.
