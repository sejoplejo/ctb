\section[The Most Common Anti-Socialist Myths]{The Most Common Anti-Socialist Myths: A Response\\\small{Written by /u/flesh\_eating\_turtle on the 12th of February 2020}}
\subsection*{Introduction}

Hello everyone.
I just wanted to provide some quick sources to refute the most common anti-socialist myths tossed around online, as a quick reference for arguments with liberals and reactionaries.
I hope you all find it useful.

\subsection*{Myth \#1: Capitalism and Liberal Democracy Are Popular}

According to a recent survey conducted by Edelman (the world's largest PR firm, based in the United States), 56\% of the world's people feel that capitalism does ``more harm than good in the world.''\footcite{john-capitalism}

In addition, a recent survey from Cambridge found that 58\% of the world's people are ``dissatisfied'' with liberal democracy.\footcite{cambridge}
This figures indicate growing global discontent with the capitalist system.

\subsection*{Myth \#2: Capitalism is Democratic}

The evidence overwhelmingly contradicts this point.
Let's take the United States as our example; according to a study from Princeton University,\footcite{princeton-theories} ``the preferences of the average American appear to have only a minuscule, near-zero, statistically non-significant impact upon public policy.''
As the study puts it:
\begin{quote}
We believe that if policy-making is dominated by powerful business organizations and a small number of affluent Americans, then America’s claims to being a democratic society are seriously threatened.
\end{quote}
In addition, a study from Northwestern University found that the wealthy ``are extremely active politically and that they are much more conservative than the American public as a whole with respect to important policies concerning taxation, economic regulation, and especially social welfare programs.''\footcite{northwestern-democracy}
They also state:
\begin{quote}
We suggest that these distinctive policy preferences may help account for why certain public policies in the United States appear to deviate from what the majority of US citizens wants the government to do.
If this is so, it raises serious issues for democratic theory.
\end{quote}
The people as a whole support significantly more left-wing policies (according to the above study, more than half of all Americans support state-run universal healthcare, wealth redistribution, and a jobs guarantee), but these policies are blocked by the ruling class.
These issues can be expected to occur in other capitalist nations as well.

\subsection*{Myth \#3: Public Ownership is Inefficient}

There is little-to-no evidence that SOEs (state-owned enterprises) are less efficient than private enterprises, given similar external conditions.
According to a study conducted at Cambridge University (put out by the United Nations Department for Economic and Social Affairs),\footcite{ha-joon} ``there is no clear systematic evidence that SOEs are burdens on the economy.''
The study further notes that ``Despite popular perception, encouraged by the business media and contemporary conventional wisdom and rhetoric, SOEs can be efficient and well-run.''
It points out:
\begin{quote}
Many countries achieved economic success with a large SOE sector... Conversely, many unsuccessful economies have small SOE sectors.
\end{quote}
A study from Stanford University's Center on Global Poverty and Development evaluated both public and private enterprises in China,\footcite{soe-performance} finding the former to be significantly more productive, even when controlling for favorable market conditions and better management:
\begin{quote}
We find that, the labor productivity and TFP of SOEs are significantly higher than private firms... Furthermore, this paper finds that, although better human capital, more market power and better management can explain partially why productivity in SOEs are higher, there remains a large share of the SOE advantage in productivity that is still left unexplained.
\end{quote}
Another study,\footcite{soe-efficiency} published in the \textit{International Journal of Production Economics}, measured the efficiency of public and private enterprises, using Spain as an example.
They found that SOEs showed similar or slightly higher efficiency relative to private enterprises:
\begin{quote}
In short, SOEs were not amongst the most inefficient in their sectors, but neither among the most efficient, showing a level of efficiency similar or slightly above the median of the efficiency of private companies... our findings would challenge the recurrent argument on the need of privatizing these companies due to their high levels of inefficiency.
\end{quote}
While some enterprises did experience an increase in efficiency after privatization, other studies have indicated that this is due to structural changes that occurred \emph{before} the privatization took place.
Even the above paper notes that ``other studies provide evidence that profitability increases \emph{before} privatization, suggesting that governments can effectively restructure companies before selling them.''
It should also be noted that in most cases (eight out of fourteen) ``differences in efficiency before and after privatization are not statistically significant.''

Other studies have supported the idea that pre-privatization restructuring is the primary factor in increased efficiency.
For example, one study,\footcite{vickers-privatization} published in the \textit{Journal of Economic Perspectives}, looks at the impact of privatization on efficiency in Britain, noting that ``the most dramatic changes have occurred in state-owned enterprises like (pre-privatization) British Steel and British Coal, where productivity gains have been massive by any standards.''

Privatization also depends fundamentally on the competence of the government which carries it out.
This presents a conundrum; according to the aforementioned UN study:\footcite{ha-joon}
\begin{quote}
At root, it appears that if a government has the capacity and capability to conduct a good privatization, it probably also has the capacity to operate good SOEs; whereas, if a government does not have the capacity to operate good SOEs, it likely also lacks the capacity to conduct a good privatization.
\end{quote}
To further complicate matters, the problems that state-owned enterprises \emph{do} have often occur in private firms as well; as the above UN study puts it:
\begin{quote}
All the key arguments against SOEs---the principal-agent problem, the free-rider problem, and the soft budget constraints---apply to large private sector firms with dispersed ownership.
\end{quote}
While public ownership is not problem-free, there is no good evidence to suggest that it is less efficient than private ownership.

\subsection*{Myth \#4: Capitalism Meets Human Needs Better Than Socialism}

Socialism has been consistently superior to capitalism in terms of meeting human needs.
A study published in the \textit{International Journal of Health Services} notes that ``contrary to dominant ideology, socialism and socialist forces have been, for the most part, better able to improve health conditions than have capitalism and capitalist forces.''\footcite{navarro-socialism}

Another study,\footcite{cereseto-economic} published in the \textit{American Journal of Public Health}, measured physical quality of life (PQL) in capitalist and socialist countries, finding that:

\begin{quote}
In 28 of 30 comparisons between countries at similar levels of economic development, socialist countries showed more favorable PQL outcomes...
Our findings indicate that countries with socialist political-economic systems can make great strides toward meeting basic human needs, even without extensive economic resources.
When much of the world's population suffers from disease, early death, malnutrition, and illiteracy, these observations take on a meaning that goes beyond cold statistics.
\end{quote}

A subsequent study,\footcite{lena-political} published in the \textit{International Journal of Health Services}, verified these results, finding that ``in general, nations with strong left-wing regimes have more favorable health outcomes (e.g., longer life expectancies and lower mortality rates) than do those with strong right-wing regimes.''

These results can be explained by referencing the aforementioned UN study;\footcite{ha-joon} as it noted:

\begin{quote}
As a ``one-dollar-one-vote'' system, markets are not likely to adequately meet the basic needs of the poor.
For example, 20 times more money is spent on research on slimming drugs than on research on malaria, a disease that kills more than a million people every year.
If we want a broad-based and politically sustainable development, we need to find mechanisms that can meet the basic needs of everyone.
\end{quote}

These facts must be taken into consideration.

\subsection*{Myth \#5: Capitalism is Eliminating Global Poverty}

According to an article by Jason Hickel (London School of Economics),\footcite{hickel-gates} global poverty is significantly higher than most people believe, due to the absurdly low poverty line used by the World Bank (\$1.90 a day). As he puts it:

\begin{quote}
It's obscenely low by any standard, and we now have piles of evidence that people living just above this line have terrible levels of malnutrition and mortality.
Earning \$2 per day doesn't mean that you're somehow suddenly free of extreme poverty.
Not by a long shot.
\end{quote}

If a more reasonable poverty standard (such as \$7.40) is used, ``we see that the number of people living under this line has increased dramatically since measurements began in 1981, reaching some 4.2 billion people today.''
It must also be noted that most actual poverty reduction since 1981 has occurred in China, which is hardly a free market society (five-year plans are still drawn up, and the state still owns most strategic industries).
As Hickel puts it:

\begin{quote}
Moreover, the few gains that have been made have virtually all happened in one place: China.
It is disingenuous, then, for the likes of Gates and Pinker to claim these gains as victories for Washington-consensus neoliberalism.
Take China out of the equation, and the numbers look even worse.
Over the four decades since 1981, not only has the number of people in poverty gone up, the proportion of people in poverty has remained stagnant at about 60\%.
It would be difficult to overstate the suffering that these numbers represent.
\end{quote}

Finally, according to a study published in the \textit{World Social and Economic Review},\footcite{incrementum} eliminating global poverty will be functionally impossible without a significant reduction in global inequality:

\begin{quote}
Poverty eradication, even at \$1.25-a-day, and especially at a poverty line which better reflects the satisfaction of basic needs, can be reconciled with global carbon constraints only by a major increase in the share of the poorest in global economic growth, far beyond what can realistically be achieved by existing instruments of development policy---that is, by effective measures to reduce global inequality.
\end{quote}

These facts make a continued capitalist model highly untenable.
